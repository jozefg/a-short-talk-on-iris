\documentclass{amsart}
\usepackage{amsmath, amssymb}
\usepackage{mathpartir}
\usepackage{natbib}
\usepackage{iris}

\newcommand{\limplies}{\to}
\newcommand{\forces}{\Vdash}
\newcommand{\defs}{\triangleq}
\newcommand{\EMP}{\textlog{Emp}}

\title{Iris, Briefly}
\author{Daniel Gratzer}
\date{\today}

\begin{document}
\maketitle

The purpose of this note is to briefly introduce those familiar with
some variant of concurrent separation
logic~\citep{OHearn:02,OHearn:07} to the Iris
logic~\citep{Jung:15,Jung:16,Krebbers:17,Jung:17}. Like O'Hearn's
logic, Iris presents a framework for reasoning about programs with
concurrency and shared state. Unlike Concurrent Separation Logic, Iris
is not a collection of inferences rules defining axiomitizing a
Hoare-triple-like judgment~\citep{Hoare:69}.

Iris comes from nearly 15 years after the first introduction of
concurrent separation logic and in that time there has been an
explosion of concurrent separation logics~\citep{Parkinson:10}. Each
new program or idiom has seemed to spawn a new concurrent separation
logic, more complicated than any that came before it, which allows the
verification of most of the previously possible programs and this new
one. This state of affairs was highly unsatisfying for a number of
reasons, chief among them being that there was no endpoint in
sight. None of these logics claimed to be a clean or complete solution
and the expectation was that in a years time the work put into this
logic must be redone in order to extend it to a new class of
programs. This constant churn also damaged the usability of each
logic, there was no point in creating tooling or any library of
verified programs in a logic that was to be obsoleted in 6 months.

Given this state of affairs, Iris is an attempt to provide a unified
foundation for a \emph{variety} of concurrent separation
logics. Rather than fixing a particular collection of rules and baking
in a few concurrency patterns Iris provides the tools needed to encode
these rules. This means that the formal definition of Iris makes no
mention of a particular form of concurrency, nor indeed a particular
programming language of any sort. There is no notion of Hoare triples
or weakest preconditions, let alone invariants or even really ghost
state. All of these notions are instead encoded which has crucially
left them available to change. Indeed, the central difference between
Iris when originally proposed~\citep{Jung:15} and the version
currently used~\citep{Jung:17} is that more and more of Iris has been
encoded rather than primitively given.

In this note we discuss the base system of Iris and how we can move
from the handful of modalities added to higher-order bunched
implication logic to a full blown separation logic. To finish, we
shall briefly discuss some of the surprising extensions made to Iris.

\section{The Base Logic of Iris}

It has long been folklore~\footnote{Listen, it's called folklore
  because I don't know who to cite for this observation} that
separation logic can been understood as an instance of bunched
implication logic. Bunched implication logic is a logic generated by a
very simple idea: rather than just having one form of expressing that
two propositions simultaneously hold, we have two. This idea
generalizes the observation in separation logic that there are two
reasonable ways that two proposition can be said to hold together on a
heap.
\begin{align*}
  h \forces P_1 \land P_2 &\defs h \forces P_1 \land h \forces P_2\\
  h \forces P_1 \sep P_2 &\defs \Exists h_1, h_2. h_1 \uplus h_2 = h
                           \land h_1 \forces P_1
                           \land h_2 \forces P_2
\end{align*}
In bunched implication logic we move away from the particulars of
owning a heap and consider having $\land$ and $\sep$ coexisting in the
same logic. As is usual, there is a corresponding nullary version of
conjuction for each of the two forms, written $\TRUE$ and $\EMP$
respectively. In order to make this applicable to separation logic, our
bunched implication logic is substructural, with the following rule
being inadmissible.
\[
  \inferrule{ }{P \vdash P \sep P}
\]
On the other hand, $\land$ behaves in the structural way one would
expect from intuitionistic logic with all four of the following rules
being derivable.
\begin{mathpar}
  \inferrule{ }{P \vdash \TRUE} \and
  \inferrule{P \vdash Q \land R}{P \vdash Q \\ P \vdash R} \and
  \inferrule{P \vdash Q \\ P \vdash R}{P \vdash Q \land R} \and
  \inferrule{ }{P \land Q \vdash Q \land P}
\end{mathpar}
The precise properties of $\sep$ are more up for debate. In
particular, it's conceivable to discuss a linear version of $\sep$
or an affine variant. To my knowledge, no separation logic makes use
of an ordered separating conjunction, but it's certainly
conceivable. For Iris we will make do with an affine separating
conjuction. This means that we will assume the following rules about
$\sep$.
\begin{mathpar}
  \inferrule{ }{\EMP \dashv\vdash \TRUE} \and
  \inferrule{ }{P \sep \EMP \dashv\vdash P} \and
  \inferrule{ }{P \sep Q \vdash Q \sep P} \and
  \inferrule{ }{(P \sep Q) \sep R \vdash P \sep (Q \sep R)} \and
  \inferrule{P_i \vdash Q_i}{P_1 \sep P_2 \vdash Q_1 \sep Q_2}
\end{mathpar}
One may reasonably observe that these rules are somewhat clunkier than
the rules governing $\land$. This is largely due to the structure of
the proof theory. There's a more natural presentation of bunched
implication logic, as given in \citet{OHearn:99}. In this presentation
the judgmental structure is richer so that $\land$ and $\sep$ are just
straightforward internalizations of ambient judgmental structure. In
particular, rather than a two-place relation between propositions as
our $\vdash$ is, this proof theory relates nested trees of
propositions (bunches) to propositions.

One outstanding question is the standing of implication in this
logic. In normal propositional
\[
  \mprset{fraction={===}}
  \inferrule{
    P_1 \vdash P_2 \limplies P_3
  }{P_1 \land P_2 \vdash P_3}
\]
Our separation conjunction though, can also be equipped with a
function space. It is defined by a similar rule
\[
  \mprset{fraction={===}}
  \inferrule{
    P_1 \vdash P_2 \wand P_3
  }{P_1 \sep P_2 \vdash P_3}
\]
With these four connectives we have the foundations of bunched
implication logic and with it, Iris. The integration of disjunction
can be done with relatively little effort. Let us now discuss the
components on top of propositional BI that Iris makes use of.

\subsection{Higher-Order Bunched Implication Logic}

The generalization of propositional bunched implication logic to
higher-order bunched implication logic is an important but still quite
simple move. In order to do this, we first must sketch out the term
structure of the language. As a matter of ergonomics the actual
iris-coq formalization uses the native Coq terms as the language
the logic is built over\footnote{This trick can by formally justified
  denotational: the model of Iris lives in the topos of trees which
  supports a full denotational model of Coq.}. On paper, this trick is
less appealing so we must be upfront about what language we are to use.

\subsection{Integrating Ghost State Parametrically}

\subsection{Guardedness and L\"ob Induction}

\subsection{The Always Modality}

\section{The Model of Iris}

\section{Building a Concurrent Separation Logic}

\section{Applications}

\section{Conclusion}


\bibliographystyle{plainnat}
\bibliography{csl}{}
\end{document}
